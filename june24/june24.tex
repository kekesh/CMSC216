\section{Monday, June 24, 2019}

\subsection{Linked Lists}
Like inner classes in Java, C structures can have pointers to structures of the same type. This allows us to define a Linked List's node as follows:


\lstset{
caption=Linked List Node}
\begin{center}
\lstinputlisting[language=c]{june24/june2401.c}\label{Command Line Parameters}
\end{center}


Note that the structure tag here is necessary here since we have a self-reference inside our definition of a node.

In order to represent a Linked List, we declare a pointer to the head by typing something like \verb!Node *head!. The pointer allows us to modify the Linked List inside of various functions.

There are two noteworthy types of Linked Lists traversal: \begin{itemize}
    \item The ``print traversal,'' which works by moving a current pointer forward after some processing 
    
    \lstset{
    caption=Print Traversal}
    \begin{center}
    \lstinputlisting[language=c]{june24/june2402.c}\label{Print Traversal}
    \end{center}
    
    \item The ``Tom and Jerry'' traversal, which works with two adjacent pointers. The left-most pointer allows us to look back and access previous elements 
    
    \lstset{
    caption=Tom and Jerry Traversal}
    \begin{center}
    \lstinputlisting[language=c]{june24/june2403.c}\label{Tom and Jerry}
    \end{center}

\end{itemize}

