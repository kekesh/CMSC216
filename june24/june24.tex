\section{Monday, June 24, 2019}

\subsection{Linked Lists}
Like inner classes in Java, C structures can have pointers to structures of the same type. This allows us to define a Linked List's node as follows:


\lstset{
caption=Linked List Node}
\begin{center}
\lstinputlisting[language=c]{june24/june2401.c}\label{Command Line Parameters}
\end{center}


Note that the structure tag inside of the typedef is necessary here since we have a self-reference inside our definition of a node.

In order to represent a Linked List, we declare a pointer to the head by typing something like \verb!Node *head!. The pointer allows us to modify the Linked List inside of various functions. So, functions that modify the actual Linked List have function prototypes that take in a double pointer Node type.  


For instance, a function which instantiates the Linked List might have header \verb!void create_list(Node **head)!, and its body would simply be \verb!*head = NULL!. 





There are two noteworthy types of Linked Lists traversal: \begin{itemize}
    \item The ``\vocab{print traversal},'' which works by moving a current pointer forward after some processing: 
    
    \lstset{
    caption=Print Traversal}
    \begin{center}
    \lstinputlisting[language=c]{june24/june2402.c}\label{Print Traversal}
    \end{center}
    
    If we need to visit every node in the list and do some processing, we perform the print traversal.
    
    
    \item The ``\vocab{Tom and Jerry traversal}'', which works with two adjacent pointers. The left-most pointer allows us to look back and access previous elements. The two pointers move in parallel. 
    
    \lstset{
    caption=Tom and Jerry Traversal}
    \begin{center}
    \lstinputlisting[language=c]{june24/june2403.c}\label{Tom and Jerry}
    \end{center}

    This traversal allows us to add a node before or after any element. 
\end{itemize}


When we construct a Linked List, we will want to add nodes one by one. Creating a node requires three key steps: \begin{enumerate}
    \item Allocate memory for the node.
    \item Store data in the node.
    \item Insert the node into the list.
\end{enumerate}

When we create a node, we write something like \verb!struct node *new_node! followed by the command \verb!new_node = malloc(sizeof(struct node)))!. Note that something like \verb!new_node = malloc(sizeof(new_node))! would be incorrect as \verb!new_node! is a pointer variable (this command would allocate $8$ bytes rather than the actual amount of space needed). 

To change the data of a node, we can use the arrow operator by doing something like \verb!new_node->data = 5!. 
