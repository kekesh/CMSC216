\section{Tuesday, May 28, 2019}
\subsection{Logistics}
\begin{enumerate}
\item All lectures are recorded and posted online.
\item No pop quizzes, no collaboration on projects.
\item Website sign-in: cmsc216/sprcoredump.
\item Office hours are immediately after class in IRB $2210$.
\item Everybody will get an Arduino -- this will be used later in the course. 
\item This class isn't curved.
\end{enumerate}

\newcommand{\ra}{\rightarrow}
\subsection{Basic Unix Commands}
Unix has lots of commands, but we want to first focus first on the ones that'll let us write and execute C programs. 
\begin{itemize}
    \item \verb!pwd! $\ra$  displays your current directory.
    \item \verb!ls! $\ra$ displays the files/directories in the current directory.
    \begin{itemize}
        \item \verb!ls -al! $\ra$ lists all of the files and directories, including hidden ones (Here, the \verb!a! flag functions to show hidden files, whereas the \verb!l! flag functions to list all entries with detailed information, like last date accessed). 
        \item \verb!ls -F! $\ra$ identifies directories by listing them with a $/$.
    \end{itemize}
    \item \verb!cd! $\ra$ change directory to the inputted parameter.
\end{itemize}
% Other easy unix stuff.  
% The 216 Public folder is where we'll be provided information
% Do work in the 216 folder.

% Set-up this:         /afs/glue/class/summer12019/cmsc/216/0101/public/bin/

\subsection{Introduction to C Programming}
In CMSC131 and 132, we learned Java. Unlike Java, C is not object-oriented; it has no concept of classes, objects, polymorphism, or inheritance. However, C can be used to implement some object-oriented concepts, like polymorphism or encapsulation. Consider the following program: 


\lstset{
caption=A First Program
}
\lstinputlisting[language=c]{may28/may2801.c}

How does this program work? 
\begin{itemize}
    \item The \verb!#include! allows the compiler to check argument types. It can compile without declaration, though the compiler will warn you. 
    \item Like Java, C provides a definition of the \verb!main()! function, where all C programs begin. 
    \item We return from \verb!main()! to end the program. For standard practice, we return $0$ to signal that everything worked out fine. 
\end{itemize}

Now, let's say we want to run this program. How can we do this? C programs need to be compiled before they can be executed. With the \verb!gcc! compiler, a very simple compilation command is \verb!gcc file.c!, from which we can run the executable by just typing $\verb!./file!$. 

Some more compilation options are summarized below (these are called \vocab{flags}): \begin{itemize}
    \item \verb!-g! enables debugging by generating and maintaining necessary symbols (e.g. line numbers) upon compilation.
    \item \verb!-Wall! warns about common things that might be a problem.
    \item \verb!-o filename! places an executable in the file name.
\end{itemize}